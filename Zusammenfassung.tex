\cite{Scheuermann:2004qf} implementiert Depth of Field mithilfe Post-Processing der Tiefenkarte und der Überblendung zwischen geblurrter und normaler Szene. Um einen schönen Unschärfeeffekt zu erzeugen wird die von uns ebenfalls verwendete Poisson-Verteilung genutzt. Durch Vergleich der Tiefenwerte sollen Samples, die zu "Leaking" von scharfen in unscharfe Objekte führen, reduziert werden.
 
\cite{Kilgard:2003dq} beschreibt kurz wie man mithilfe der Sprache Cg Fragment- und Vertex-Shader schreiben kann.

\cite{Earll-Hammon:2007oq} erläutert den Algorithmus der bei COD4 genutzt wurde.Beschreibt vorher vier verschiedene Versuche den Effekt umzusetzen. Algorithmus lässt vordergrundobjekte in Hintergrundobjekte bluten.

\cite{Dudash:2004fk} stellt einen Echtzeit Tiefenschärfe-Algorithmus vor, der unserem vom Ablauf her sehr ähnelt.

\cite{Sungkil-Lee:2009zr} kann mit nur einem Rendering-Durchlauf Tiefenschärfe-Effekte qualitativ ähnliche Effekte wie Multiview-Verfahren liefern. Die Probleme eines Singleview-Verfahrens werden hier allerdings durch eine Aufteilung des Bildes umgangen. Die erstellten Layer des Bildes sind in verschiedene Tiefenbereiche aufgeteilt, die unterschiedlich stark geblurrt werden. Das Verfahren liefert gute, artefaktfreie, Ergebnisse und ist hoch skalierbar.

\cite{Michael-Kass:2006uq} ist das Paper der Pixar Studios. Der Algorithmus ermöglicht Previews von Animationsfilmen mithilfe der Hitzeformel. Algorithmen, wie sie in Spielen genutzt werden sind für den Film unbrauchbar. In Echtzeit nur mit begrenzter Auflösung lauffähig.

\cite{Marcelo-Bertalmio:2004fk} nutzt eine Differentialgleichung um das 2D-Bild mithilfe der Tiefenwerte entsprechend zu blurren. Hier kommt ebenfalls die Hitzegleichung zum Einsatz.

\cite{Guennadi-Riguer:2003kx} erklärt den Depth of Field-Effekt und seine Umsetzung, ähnlich der unseren, allerdings mithilfe von Gauss

\cite{Sungkil-Lee:2008bh} erzeugt den Tiefenschärfe-Effekt durch Nutzung verschiedener MipMaps. Die Eliminierung der Artefakte soll mittels eines Gauss-Filters realisiert werden.

\cite{Sungkil-Lee:2010ve} beschreibt ein neues Rendering-System für Defokus- und Blureffekte. Bietet bessere Präzision als vergleichbare Echtzeit-Verfahren. Ermöglicht Nutzung anderer Linsenmodelle.